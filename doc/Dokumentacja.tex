\documentclass[11pt,a4paper]{article}

\usepackage[T1]{fontenc}
\usepackage[utf8]{inputenc}
\usepackage[polish]{babel}

\setlength\parindent{0px}

\title{\vspace{-2.0cm}\textbf{Projekt: rendering obrazów - dokumentacja}}
\author{Michał Jaworski}
\begin{document}
\maketitle
\section{Używanie programu}
Program należy wywołać z wiersza poleceń podając jako argument plik zawierający opis sceny. Po uruchomieniu, jeśli plik został prawidłowo wczytany, program wyświetli wygenerowany obraz oraz zapisze go w formacie BMP.
\section{Format opisu sceny}
Opis sceny wczytywany jest z pliku tekstowego w opisanym niżej formacie.
\begin{itemize}
\item Wiersze rozpoczynające się od znaku \# są traktowane jako komentarze i ignorowane przez parser.
\item Wielkość liter w słowach kluczowych nie ma znaczenia
\item Wektory oraz punkty w przestrzeni trójwymiarowej przedstawione są jako trzy liczby rzeczywiste oddzielone białymi znakami
\item W podobny sposób reprezentowane są kolory w postaci RGB, kolejne liczby oznaczają wartości na odpowiednich kanałach
\item W parserze zdefiniowano również stałe reprezentujące podstawowe kolory. Są to: \textit{black}, \textit{white}, \textit{red}, \textit{green}, \textit{blue}, \textit{cyan}, \textit{magenta} oraz \textit{yellow}
\end{itemize}
Plik tekstowy opisujący scenę składa się z następujących części:
\begin{enumerate}
\item\textbf{Nagłówek} (obowiązkowy), każda z poniższych informacji poprzedzona jest odpowiednim słowem kluczowym:
\begin{itemize}
\item\textit{imWidth} oraz \textit{imWeight}: liczby całkowite reprezentujące odpowiednio szerokość oraz wysokość obrazka w pikselach, podanych jako liczby całkowite
\item\textit{canvWidth} oraz \textit{canvHeight}: liczby rzeczywiste reprezentujące odpowiednio szerokość oraz wysokość prostokąta, przez który obserwowana jest scena. podanych jako liczby całkowite
\item\textit{depth}: liczba rzeczywista reprezentująca odległość ogniska od prostokąta
\item\textit{bgcolor}: opcjonalny parametr oznaczający kolor tła. Domyślnie jest to kolor czarny
\item\textit{raydepth}: opcjonalny parametr określający maksymalną głębokość rekursji podczas śledzenia promieni. Domyślną wartością jest 4
\end{itemize}
\item\textbf{Źródła światła} (opcjonalne). Lista źródeł światła rozpoczynająca się słowem kluczowym \textit{lights}. Dostępne są następujące źródła światła:
\begin{itemize}
\item\textit{directional i c d}: kierunkowe źródło światła o intensywności wyznaczonej przez liczbę rzeczywistą \textit{i}, o kolorze \textit{c}, świecące w kierunku wskazywanym przez wektor \textit{d}
\item\textit{spherical i c x}: punktowe źródło światła o intensywności \textit{i}, kolorze \textit{c}, znajdujące się w punkcie \textit{x}
\end{itemize}
\item\textbf{Obiekty} (opcjonalne). Lista znajdujących się na scenie obiektów. Każdy obiekt przedstawiony jest w postaci \textit{kształt powierzchnia}. Program udostępnia następujące kształty:
\begin{itemize}
\item\textit{sphere x r}: kula o środku \textit{x} i promieniu \textit{r}
\item\textit{plane x d}: płaszczyzna zawierająca punkt \textit{x} o wektorze normalnym równoległym do wektora \textit{d}
\end{itemize}
oraz następujące rodzaje powierzchni:
\begin{itemize}
\item\textit{diffusive c}: powierzchnia rozpraszająca światło, w kolorze \textit{c}
\item\textit{reflective}: powierzchnia odbijająca światło
\item\textit{luminous c}: powierzchnia świecąca własnym światłem, w kolorze \textit{c}
\item\textit{mixed $t_{1}\ s_{1}\ [t_{2}]\ [s_{2}]$ ...}: powierzchnia mieszana, gdzie wartości \textit{$t_{i}$} wyznaczają proporcje, a \textit{$s_{i}$} to mieszane rodzaje powierzchni
\end{itemize}
\end{enumerate}
\section{Dokumentacja kodu}
\subsection{Datatypes.hs}
\textit{Vector} - typ danych reprezentujący wektory trójwymiarowe
\end{document}