\documentclass[11pt,a4paper]{article}

\usepackage[T1]{fontenc}
\usepackage[utf8]{inputenc}
%\usepackage{indentfirst}
\usepackage[polish]{babel}

\title{\vspace{-2.0cm}\textbf{Projekt: rendering obrazów - raport}}
\author{Michał Jaworski}

\begin{document}
\maketitle
\section{Reprezentacja sceny}
Istotnym problemem, jaki napotkałem podczas pisania programu, było zaprojektowanie odpowiedniego formatu tekstowego do reprezentacji sceny. Format ten powinien być przede wszystkim czytelny. Musi on także zapewniać możliwość łatwego rozszerzania, w przypadku dodania do programu nowych obiektów, rodzajów powierzchni itp.

Opracowany przeze mnie format został szczegółowo omówiony w dokumentacji projektu. W celu ułatwienia procesu parsowania użyłem biblioteki Parsec, która dostarcza wielu funkcji pomocnych podczas tworzenia parserów.
\section{Typy danych}
Istotnym problemem podczas pracy nad projektem było stworzenie odpowiednich typów danych oraz klas typów. W moim programie zdecydowałem się na reprezentację większości typów związanych z opisem sceny jako abstrakcyjnych typów danych. Umożliwia to reprezentację zawartości sceny w postaci list. Z kolei kolory stanowią w moim programie klasę typów. Umożliwia to łatwe dodanie kolejnych sposobów ich reprezentacji oraz wyrenderowanie sceny z dowolnie wybranym z nich.
\section{Umiejscowienie kamery}
Kolejnym problemem, jaki napotkałem, było odpowiednie ustawienie kamery. Aby maksymalnie uprościć obliczenia, jak również zapewnić użytkownikowi łatwe rozmieszczanie obiektów na scenie zdecydowałem się rozwiązać ten problem w sposób następujący:
\begin{itemize}
\item Kamera patrzy w kierunku wyznaczonym przez wektor $(0, 0, 1)$.
\item Środek prostokąta widoku znajduje się w punkcie $(0, 0, 0)$.
\item Ognisko widoku znajduje się w punkcie $(0, 0, -\textit{depth})$.
\end{itemize}
\section{Odbijanie promieni}
Implementując powierzchnie odbijające światło, uczyniłem śledzenie promieni procesem rekurencyjnym. Dzięki temu, aby wyznaczyć kolor danego punktu powierzchni lustrzanej wystarczy rekurencyjnie wyznaczyć kolor obiektu, którego obraz odbija się w badanym punkcie, rekurencyjnie śledząc odpowiednio wyznaczony promień odbity. Niestety, takie podejście dopuszcza możliwość nieskończonej rekursji (np. gdy scena składa się z dwóch równoległych, lustrzanych płaszczyzn). Aby temu zapobiec ograniczyłem głębokość rekurencyjnych wywołań funkcji. Domyślnie dopuszczalne są maksymalnie 4 wywołania, użytkownik może jednak ustawić własną wartość w nagłówku pliku opisującego scenę. W przypadku przekroczenia maksymalnej głębokości rekursji zwracany jest kolor tła (tzn. promień uznaje się za niekolidujący z żadnym obiektem sceny).
\end{document}